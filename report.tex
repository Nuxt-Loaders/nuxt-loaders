\documentclass[a4paper,12pt]{article}

\usepackage{graphicx}
\usepackage{hyperref}
\usepackage[a4paper, margin=1in]{geometry}
\usepackage{fancyhdr}
\usepackage{times}
\usepackage{enumitem}
\usepackage{xcolor}

\pagestyle{fancy}
\fancyhf{}
\rfoot{\thepage \ | P a g e}
\renewcommand{\headrulewidth}{0pt}

\begin{document}

% ---------- COVER PAGE ----------
\begin{titlepage}
    \centering
    \vspace{2cm}

    % Logo from first document
    \includegraphics[width=0.25\textwidth]{logo.png}\\[2cm]

    {\Huge \textbf{ADDIS ABABA SCIENCE AND}}\\[1cm]
        {\Huge \textbf{TECHNOLOGY UNIVERSITY}}\\[1cm]
    {\Large \textbf{COLLEGE OF ENGINEERING}}\\[0.3cm]
    {\Large \textbf{DEPARTMENT OF SOFTWARE ENGINEERING}}\\[0.8cm]

    {\Large \textbf{Open Source Software Development (OSSP)}}\\[0.6cm]

    {\Large \textbf{Nuxt Loaders: A Highly Optimized and Customizable Loading Engine for Nuxt.js}}\\[1cm]

    \textbf{\large Section B}\\[1cm]

    % Students list
    \textbf{\large Group Members}\\[0.3cm]
    \large
    Haileab Tesfaye (ETS0714/14)\\
    Ephrem Mandefro (ETS0536/14)\\
    Haileyesus Asrat (ETS0717/14)\\
    Henok Tademe (ETS0775/14)\\[1.2cm]

    \textbf{\large Submitted to: Inst. Biniam B. (MSc)}\\[1.5cm]

    {\large \today}

    \vfill
\end{titlepage}

% Reset page number
\setcounter{page}{1}
\clearpage

% IEEE-like formatting for the paper
\pagestyle{plain}
\setlength{\parindent}{0pt}
\setlength{\parskip}{6pt}

\begin{center}
    {\Large \textbf{Nuxt Loaders: A Highly Optimized and Customizable Loading Engine for Nuxt.js}}
    
    
    \textit{Department of Software Engineering, College of Engineering}\\
    \textit{Addis Ababa Science and Technology University, Addis Ababa, Ethiopia}
\end{center}

\vspace{1cm}

\textbf{Abstract---}This paper presents the design and development of \textit{Nuxt Loaders}, an open-source, highly optimized loading engine built for Nuxt.js applications. The project focuses on improving performance, customization, and developer experience by providing flexible loader components suitable for both landing pages and full-scale administrative dashboards. Emphasis is placed on modular architecture, Nuxt adapter integration, and community-driven contribution. The system addresses the technical gap in modern Nuxt applications where loading states often cause layout shifts and poor user retention.

\vspace{0.5cm}

\textbf{Keywords---}Nuxt.js, Loaders, Frontend Optimization, UI/UX, Open Source Software, Human-Computer Interaction

\vspace{1cm}

\section{Project Goals}
The primary goal of this project is to develop a professional and optimized loading system tailored for the Nuxt.js ecosystem. In the context of Human-Computer Interaction (HCI), providing immediate visual feedback during network-bound operations is critical for maintaining user engagement and reducing perceived latency. The specific objectives include:
\begin{itemize}[leftmargin=*, label=\small$\Phi$]
    \item \textbf{High Optimization}: Building loaders using CSS hardware acceleration and minimal JavaScript to ensure zero performance impact on the main thread.
    \item \textbf{Universal Support}: Providing templates optimized for diverse layouts, specifically distinctions between High-Impact Home Pages and Functional Admin Dashboards.
    \item \textbf{Seamless Integration}: Leveraging Nuxt's module system (Adapters) to allow "single-line" integration without manual component registration.
    \item \textbf{Extensibility and Contribution}: Structuring the codebase to be "open" for community developers, allowing them to contribute new loaders easily.
    \item \textbf{Documentation Excellence}: Providing clear, readable, and developer-friendly documentation to lower the barrier to entry for users and contributors.
\end{itemize}

\section{Development Process}
The development lifecycle of \textit{Nuxt Loaders} followed a rigorous engineering path, from conceptual research to branding and community outreach.

\subsection{Research and Investigation}
Initial research involved a deep analysis of existing loading solutions in the Nuxt environment. Most current implementations were either overly generic (leading to layout shifts) or too complex to customize without deep CSS knowledge. We investigated the \textit{First Contentful Paint} (FCP) and \textit{Cumulative Layout Shift} (CLS) impacts of different loader strategies to establish our baseline requirements.

\subsection{Nuxt Adapter Integration}
A major technical milestone was the integration with the Nuxt module system. Understanding the Nuxt "adapter" logic—specifically how to dynamically generate virtual templates and register components at build time—was initially challenging due to the sparse documentation for advanced module features.
\textbf{Solution:} Through a deep investigation of the \texttt{@nuxt/kit} source code and experimentation with Nitro server engine hooks, we developed a robust registration engine that automatically detects and injects loader components into the user's application.

\subsection{Loader Design and Animation}
Each loader was designed with specific HCI principles in mind. For instance, the \texttt{CircularLoader} uses a pulsing easing function to reduce user anxiety during wait times, while the \texttt{PulseRailLoader} provides a sense of horizontal progress.
\textbf{Branding Elements}: We focused on creating a cohesive visual identity. This included designing a project logo (placeholder in cover page) and developing graphic content intended for social media promotion to attract contributors.

\section{Challenges Faced and Solutions}
\subsection{Limited Adapter Documentation}
\textbf{Challenge}: The specialized nature of Nuxt's internal plugin registration made it difficult to implement route-based loader switching effortlessly.
\textbf{Solution}: We spent significant time studying Nuxt core code and benchmarking community-refined modules. This led to the development of our \texttt{routeRules} parser, which allows developers to map loaders to specific URL patterns.

\subsection{Performance Optimization}
\textbf{Challenge}: Animated loaders can often cause CPU spikes on low-end mobile devices.
\textbf{Solution}: We opted for lightweight, purely CSS-driven animations (where possible) and used Vue 3's \texttt{shallowRef} for state management to avoid unnecessary reactivity overhead.

\section{Community Engagement Strategies}
Engagement was not an afterthought but a core part of the development process.
\begin{itemize}[leftmargin=*, label=\small$\Phi$]
    \item \textbf{Social Communities}: The project was shared across Ethiopian software development channels and global Telegram groups.
    \item \textbf{CodeNight Collaboration}: We actively participated in the CodeNight Telegram community, seeking feedback from peer developers. This interaction provided critical insights into developer needs, leading to the creation of the "contribution adapters"—simplified templates that make adding new loaders a 5-minute task.
    \item \textbf{Open Source Onboarding}: By providing clear \texttt{CONTRIBUTING.md} and \texttt{CHANGELOG.md} files, we established a professional standard that motivates developers to fork and contribute.
\end{itemize}

\section{Overall Impact and Feedback Received}
\textit{Nuxt Loaders} has received highly positive initial feedback.
\begin{itemize}[leftmargin=*, label=\small$\Phi$]
    \item \textbf{Developer Feedback}: Users highlighted the "Zero Configuration" aspect and the quality of the built-in templates.
    \item \textbf{Growth Opportunities}: While the technical foundation is robust, we recognize that widespread adoption requires ongoing promotion. Future efforts include making the project "viral" within the Nuxt ecosystem through community-driven template additions.
\end{itemize}

\section{References}
\begin{itemize}
 \item GitHub Repository: \url{https://github.com/EpphremM/nuxt-loaders.git}
    \item Documentation: \url{http://nuxt-loaders.vercel.app}
    \item \textbf{Nuxt.js Official Docs}: \url{https://nuxt.com/docs}
\end{itemize}

\end{document}
